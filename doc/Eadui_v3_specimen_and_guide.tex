%&program=xelatex
%&encoding=UTF-8 Unicode
\ifdefined\suppressfontnotfounderror
  \expandafter\let\csname xetex_suppressfontnotfounderror:D\endcsname
    \suppressfontnotfounderror
\else
  \expandafter\let\csname xetex_suppressfontnotfounderror:D\endcsname
    \luatexsuppressfontnotfounderror
\fi
\documentclass[11pt,letterpaper,twoside,openany]{book}
%
%
\usepackage{fontspec}
\defaultfontfeatures{Ligatures=TeX,Contextual=on}
\setmainfont{Eadui}[Renderer=Harfbuzz]
\newfontface{\irrm}{Irradiant-Regular.otf}[Renderer=Harfbuzz]
\newfontface{\irsc}{Irradiant-Regular.otf}[Renderer=Harfbuzz,Letters=SmallCaps]
\newfontface{\irit}{Irradiant-Italic.otf}[Renderer=Harfbuzz]
%
%
\usepackage[twoside,text={3.8in,502pt},headheight=12pt,headsep=12pt]{geometry}
\usepackage{xltxtra}
\usepackage{fancyhdr}
\usepackage{lettrine}
\usepackage[first=0, last=10, seed=240]{lcg}
\usepackage{wallpaper}

\usepackage{wallpaper}
\newopentypefeature{NarrowT}{on}{+ss03}
%
% Some mnemonics to help select Eadui's features.
\newopentypefeature{SentenceCaps}{on}{+ss11}
\newopentypefeature{Alty}{on}{+ss10}
\newopentypefeature{AltCaps}{on}{+ss12}
\newopentypefeature{AltDiacritics}{on}{+ss13}
\newopentypefeature{LargeMinuscules}{on}{+ss14}
\newopentypefeature{Contextual}{on}{+calt}

% Compactly dealing with character variant features
\newcommand{\cvd}[3][0]{{\addfontfeature{CharacterVariant=#2:#1}#3}}

% End of Eadui mnemonics
%
% N.B. contextuals (calt) should always be on!
%
% Some convenient commands for working with Eadui.

\newcommand{\rubric}[1]{{\addfontfeature{StylisticSet=5}#1}}
\newcommand{\redtext}[1]{{\addfontfeature{StylisticSet=17}#1}}
\newcommand{\goldtext}[1]{{\addfontfeature{StylisticSet=19}#1}}
\newcommand{\greentext}[1]{{\addfontfeature{StylisticSet=20}#1}}
\newcommand{\bluetext}[1]{{\addfontfeature{StylisticSet=18}#1}}
\newcommand{\redfilltext}[1]{{\addfontfeature{StylisticSet=1}#1}}
\newcommand{\bluefilltext}[1]{{\addfontfeature{StylisticSet=2}#1}}
\newcommand{\yellowfilltext}[1]{{\addfontfeature{StylisticSet=3}#1}}
\newcommand{\greenfilltext}[1]{{\addfontfeature{StylisticSet=4}#1}}

\newcommand{\atc}[1]{{\addfontfeatures{Style=TitlingCaps,AltCaps=on}{#1}}}
\newcommand{\ac}[1]{{\addfontfeatures{AltCaps=on}{#1}}}
\newcommand{\salt}[1]{{\addfontfeatures{Alternate=0}{#1}}}
\newcommand{\saltb}[1]{{\addfontfeatures{Alternate=1}{#1}}}
\newcommand{\altd}[1]{{\addfontfeatures{AltDiacritics=on}{#1}}}
\newcommand{\dlig}[1]{{\addfontfeatures{Ligatures=Rare}{#1}}}
\newcommand{\hlig}[1]{{\addfontfeatures{Ligatures=Historic}{#1}}}
\newcommand{\scap}[1]{{\addfontfeatures{StylisticSet=11}#1}}
\newcommand{\swash}[1]{{\addfontfeatures{Style=Swash}{#1}}}
\newcommand{\hist}[1]{{\addfontfeature{Style=Historic}{#1}}}
\newcommand{\lmin}[1]{{\addfontfeature{LargeMinuscules=on}{#1}}}
\newcommand{\eadui}[1]{{\fontspec{Eadui} #1}}
% End of Eadui commands
%
\newcommand{\opentype}{Open\kern-2pt Type}
\newcommand{\xetex}{X\kern-0.5pt e\kern-2pt Te\kern-1pt X}
\renewcommand{\DefaultLraise}{0.2}
\renewcommand{\DefaultLhang}{0.5}
\renewcommand{\DefaultLoversize}{0.1}
%\renewcommand{\DefaultFindent}{0.5em}
\renewcommand{\DefaultNindent}{0em}
\newenvironment{junicode}{\fontspec[NarrowT=on]{Junicode}\normalsize}{}
%
% A "medieval" environment turns on historical forms.
\newenvironment{medieval}{\addfontfeatures{Style=Historic,Alty=on}}{}
%
%
\newcommand{\filltext}[3]{{\fontspec{EaduiFill}\color{#1}#3}%
\llap{\fontspec{Eadui}\color{#2}#3}}
\newcommand{\filltitl}[3]{{\fontspec[Style=TitlingCaps]{EaduiFill}\color{#1}#3}%
\llap{\fontspec[Style=TitlingCaps]{Eadui}\color{#2}#3}}
\newcommand{\junb}[1]{{\fontspec{Junicode}\bfseries{#1}}}
\newcommand{\jun}[1]{{\fontspec{Junicode}{#1}}}
\newcommand{\editem}[1]{{\fontspec{Junicode}\color{myRed}#1}}
\frenchspacing
\linespread{1.04}
\tolerance=1000
\renewcommand{\headrulewidth}{0pt}
\pagestyle{fancy}
\fancyhead{}
\fancyfoot{}


\begin{document}
%\TileSquareWallPaper{5}{parch-light.jpg}
\CenterWallPaper{0.9}{blank_ms_page_1-bright.jpg}
%\TileSquareWallPaper{6}{parch-light.jpg}
%\color{stdText}%\pagecolor{bgColor}
\Large\lettrine[lines=3,lraise=0,lhang=1]{\addfontfeature{StylisticSet=18}E}{}%
%{\textcolor{myBlue}{}}
{\addfontfeatures{Style=TitlingCaps,StylisticSet=5}adui is
  named for a scribe who worked at} Christ Church, Canterbury, in the
first half of the eleventh century and signed himself
“\scap{Ead\-uuius cognomento Basan}.”\footnote{\irit Eaduuius \irrm is a Latinized spelling
of the Old English name \irit Eadwig\irrm, and \irit Eadui \irrm is a modern respelling of the Latinized
name.}
\greentext{T}his \scap{Eadui Basan}
was a leading practitioner of the scribal hand known to paleographers
as style \atc{iv} \scap{E}ng\-lish caroline
minuscule. \redtext{L}ike caroline minuscules
generally, this one is notable for its legibility; and \scap{E}adui's
work, at its best, possesses a formal beauty that is matched by few
scribes of his time.

\lettrine[lines=2,lraise=0]{\goldtext{U}}{%
  \greentext{nlike}} most “medieval” fonts, this one aims not so much to convey
a medieval feeling as to reproduce a great scribe's style with
fidelity, avoiding the machined look of the digital font in favor of
the irregular and uneven look of a handmade thing. Lines are rarely straight
and often a little wavery, serifs
differ in shape from one glyph to another, letters do not always sit
cleanly on the baseline, and some of the most common letters vary by contex.\

\lettrine[lines=2,lraise=0]{\salt{\redtext{E}}}{\goldtext{adui}}
is a color font---one in which the colors have been chosen by the designer.
You can choose any of six different
colors for text, from the default dark brown to a vivid red for rubrics.
A seventh color is a wildcard---it can be whatever the user wants. In addition,
most letters can be filled with one of four different colors. All of the
font's colors are based on the colors of Eadui's manuscripts.

\lettrine[lines=2,lraise=0]{\greentext{T}}{\redtext{he font}}
can be used in either of two modes: modern or historical. Historical mode
emulates the usage of Eadui and other medieval scribes, with (for example)
round r (ꝛ), where the context demands it, and long s (\hist{s}). For an
even more authentic look, select historical ligatures, emulating
Eadui's ligatures, which could be eccentric, even for his time. The
font's default mode is modern, designed to be legible for modern readers.

\lettrine[lines=2,lraise=0]{\bluetext{E}}{\redtext{adui}}
the font (like the scribe)
  employs a variety of capitals for various purposes. The default
  capitals are those he used for what are now called
  “drop caps” (though they can also be used to begin sentences and proper
  names). Titling Caps are for titles and passages in all caps,
  and Sentence Caps (often identical to Titling Caps) were
  used by Eadui to begin sentences. All of Eadui's capabilities can
  be selected with OpenType features, listed at the bottom of this
  document.

\lettrine[lines=2,lraise=0]{\redtext{T}}{\goldtext{he colors}}
of the default version of Eadui are meant to be used on a light background. If
you want to use Eadui on a dark background, make sure to install the
file with “Dark” in the name, and select the “Medium Dark” style in
your application. Alternatively, if you are creating a web page, mount just
one of Eadui's font files on the server and use
the CSS “color-palette” property to select the light (0) or the dark
(1) palette.


\newpage
\noindent Historic mode (hist):\\[1ex]

\begin{medieval}
\huge\lettrine[lines=3,lraise=0,lhang=1]{\bluetext{B}}%
{}{\scshape \rubric{eat᷒ uir qͥ non abiit in
consilio impiorum⹎}} \& in uia peccatorum non stetit⹎ \cvd{64}{\&} in cathedra
pestilentiae n̄ \kern-2pt sedi\swash{t}.

\lettrine[lhang=1]{\redtext{S}\hspace{3pt}}{}ed in lege dn̄i fuit uoluntas eius⹎
\cvd[2]{64}{\&} in lege eius meditabitur die ac no\hlig{ct}\swash{e}.

\lettrine[lhang=1]{\goldtext{E}\hspace{3pt}}{}t erit tanquam lignum quod
plantatum est secus decursus aquarum⹎ \cvd[2]{34}{q}uod fru\hlig{ct}um suum dabit in
tempore suo.

\lettrine[lhang=1]{\addfontfeatures{Ligatures=Rare}\greentext{ET}\hspace{3pt}}{}foliū
ei᷒ \kern-1pt non defluet⹎ \cvd[1]{64}{\&} omnia quęcumq: faciet prosperabu\swash{n}tᷣ.

\lettrine[lhang=1]{\redtext{N}\hspace{3pt}}{}on sic impi\swash{i} \salt{n}on sic⹎ \cvd{38}{s}ed
tanquam pul\-uis quem ꝓicit uent᷒ a facie terr\altd{ę}.

\lettrine[lhang=1,nindent=0.2em]{\bluetext{I}\hspace{3pt}}{}deo non resurgu\dlig{nt} impii in
iudicio⹎ \cvd[3]{28}{n}eq: peccat\hlig{or}es in consilio iustorum.

\lettrine[lhang=1]{\addfontfeatures{AltCaps=on}\goldtext{Q}\hspace{3pt}}{}\altd{m̄} nouit
dn̄s uiam iustorum⹎ \cvd[3]{64}{\&} it͛ impioꝝ ꝑibi\swash{t}.

\end{medieval}


% The next two sections are causing crash with mysterious error message:
%! error:  (linebreak): invalid node with type whatsit and subtype 16 found in d
%iscretionary
%!  ==> Fatal error occurred, no output PDF file produced!%


\newpage
\noindent Historical forms and ligatures (hist, hlig):\\[1ex]

\begin{medieval}
\Large\noindent{\addfontfeatures{Style=TitlingCaps,StylisticSet=5}%
   \salt{G}\kern+1.5pt loriosissi\salt{m}o regi C\salt{e}oluulfo Baeda
   f\salt{a}mul᷒ Christi et presbyter}

\addfontfeatures{Ligatures=Historic,SentenceCaps=on}
%\lettrine[lines=3,lraise=0]{H}{istoriam} 
\noindent\redtext{{\huge H}istoriam} gentis Anglorum ecclesiasticam, quam nuper edi\-deram,
libentissime tibi desideranti, rex, et prius ad legendum ac probandum
transmisi, et nunc ad transscribendum ac plenius ex tempore meditandum
retransmitto; satisque studium tuae sinceritatis amplector, quo non
solum audiendis scripturae sanctae uerbis aurem sedulus accommodas,
uerum etiam noscendis priorum gestis siue dictis, et maxime nostrae
gentis uirorum inlustrium, curam uigilanter impendis. Siue enim
historia de bonis bona referat, ad imitandum bonum auditor sollicitus
instigatur; seu mala commemoret de prauis, nihilominus religiosus ac
pius auditor siue lector deuitando quod noxium est ac peruersum, ipse
sollertius ad exsequenda ea, quae bona ac Deo digna esse cognouerit,
accenditur. Quod ipsum tu quoque uigilantissime deprehendens,
historiam memoratam in notitiam tibi simul et eis, quibus te regendis
diuina praefecit auctoritas, ob generalis curam salutis latius
propalari desideras. Vt autem in his quae scripsi ƚ tibi, ƚ
ceteris auditoribus siue lectoribus huius historiae occasionem
dubitandi subtraham, quibus haec maxime auctoribus didicerim, breuiter
intimare curabo.
\end{medieval}


\newpage
\noindent Rustic capitals (smcp):\\[1ex]

\begin{medieval}
%\lettrine[lines=3,lraise=0]{B}{rittania oceani insula,}
\scshape
\noindent%
\goldtext{{\huge B}rittania} oceani insula,
\addfontfeatures{AltCaps=on} \ cui quondā
\salt{Al}bion nomen fuit, inter septentrionē \kern-2pt \& occidentē locata
est, Germanię, Gallię, Hispanię, maximis \salt{E}uropę partib᷒, multo
interuallo aduersa. Quae per miliapas\salt{s}uū .dccc. in Boreā
longa, latitudinis hab\& milia .cc., exceptis dumtaxat pro\-lixiorib᷒
diuersorū promontoriorū tractib᷒, quibus efficitᷣ, ut circuit᷒ ei᷒
quadragies octies .lxxv. milia conpleat. Habet a meridie Galliā
Belgicā, cui᷒ proximū lit᷒ transmeantib᷒ aperit ciuitas, quae dicitᷣ
Rutubi port᷒, a gente \salt{A}nglorū nunc corrupte Reptacaestir uocata,
interposito mari a Ges\salt{s}oriaco Morynorū gentis litore proximo,
traiectu miliū .l., siue, ut quidā scripsere, stadiorum .ccccl. A
tergo autem, unde Oceano infinito patet, Orcadas insulas habet. Opima
frugib᷒ atque arborib᷒ insula, \& alendis apta pecorib᷒ ac
iumentis; uineas \&iam quibusdā in locis germinans; sed \& auiū
ferax terra mariq: generis diuersi; fluuiis quoq: multū piscosis ac
fontib᷒ praeclara copiosis, \& quidē praecipue issicio abundat, \&
anguilla.
\end{medieval}

\newpage
\noindent Modern languages:\\[1ex]

\huge%
\lettrine[lines=3,lraise=0,lhang=0.5,findent=-25pt,nindent=25pt]%
{\greentext{H}}{\greentext{eureux}}
l'homme qui ne marche pas selon le conseil des méchants,
\bluetext{Q}ui
ne s'arrête pas sur la voie des pécheurs,
\greentext{E}t qui ne s'assied pas en
compagnie des moqueurs,

\lettrine[lines=2,lraise=0,lhang=1]{\bluetext{M}}%
{\bluetext{ais}}
qui trouve son plaisir dans la loi de l'Éternel,
\redtext{E}t qui la médite
jour et nuit!

\lettrine[lines=2,lraise=0,lhang=1]{\redtext{I}\kern+5pt}%
{\redtext{l est}}
comme un arbre planté près d'un cou\-rant d'eau,
\redtext{\salt{Q}}ui donne son
fruit en sa saison,
\bluetext{E}t dont le feuillage ne se flétrit point:
\goldtext{T}out ce
qu'il fait lui réussit.

\lettrine[lines=2,lraise=0,lhang=1]{\redtext{I}\kern+5pt}%
{\redtext{l n'en}}
est pas ainsi des méchants:
\bluetext{I}\kern+1pt ls sont comme la paille que le
vent dissipe.

\lettrine[lines=2,lraise=0,lhang=1]{\goldtext{C}}%
{\goldtext{'est}}
pourquoi les méchants ne résistent pas au jour du jugement,
\bluetext{N}i
les pécheurs dans l'assemblée des justes;

\lettrine[lines=2,lraise=0,lhang=1]{\greentext{C}}%
{\greentext{ar}}
l'Éternel connaît la voie des justes,
\redtext{E}t la voie des pécheurs mène
à la ruine.

\newpage

{\addfontfeature{Language=Polish,StylisticSet=13}
\lettrine[lines=3,lraise=0,lhang=0.5,findent=-10pt,nindent=10pt]%
{\redtext{B}}{\bluetext{łogosławiony mąż}}, który nie chodzi w radzie 
niepobożnych, a na drodze grzesz\-nych nie stoi,
i na stolicy naś\-niewców nie siedzi;

\lettrine[lines=2,lraise=0,lhang=1,findent=-10pt,nindent=10pt]{\bluetext{A}}{\goldtext{le}}
w zakonie \redfilltext{Pańskim} jest kochanie jego, 
a w zakonie jego rozmyśla we dnie i w nocy.

\lettrine[lines=2,lraise=0,lhang=1,findent=-10pt,nindent=10pt]{\greentext{A}}{\goldtext{lbowiem}}
będzie jako drzewo nad strumieniem wód w sadzone, które owoc swój wydaje czasu swego, 
a liść jego nie opada; i wszystko, cokolwiek czynić będzie, posz\-części się.

\lettrine[lines=2,lraise=0,lhang=1,findent=-5pt,nindent=5pt]{\goldtext{L}}{\bluetext{ecz}}
nie tak niepobożni; ale są jako plewa, którą wiatr rozmiata.

\lettrine[lines=2,lraise=0,lhang=1]{\redtext{P}}{\bluetext{rzetoż}}
się niepobożni na sądzie nie ostoją, ani grzesznicy w zgromadzeniu sprawiedliwych.

\lettrine[lines=2,lraise=0,lhang=1]{\greentext{A}}{\goldtext{lbowiem}} zna \bluefilltext{Pan}
drogę sprawiedliwych; ale droga niepobożnych zginie.}

\newpage

\huge\lettrine[lines=2,lraise=0,lhang=1]%
{\salt{\redtext{D}}}{\rubric{er Herr}} ist mein
\bluefilltext{\salt{H}}irte, mir wird nichts mangeln.

\lettrine[lines=2,lraise=0,lhang=1]{\bluetext{\salt{E}}}%
{\greentext{r}} weidet mich auf einer grünen \redfilltext{Aue} und
führet mich zum frischen \yellowfilltext{Wasser}.

\lettrine[lines=2,lraise=0,lhang=1]%
{\goldtext{E}}{\redtext{r}}
erquicket meine \greenfilltext{Seele}. \redfilltext{Er} führet mich auf rechter \bluefilltext{Straße} um seines
\yellowfilltext{Namens} willen.

\lettrine[lines=2,lraise=0,lhang=1]{\greentext{U}}%
{\goldtext{nd}} ob ich schon wanderte im finstern \bluefilltext{Tal},
fürchte ich kein \greenfilltext{Unglück};
denn du bist bei mir, dein \yellowfilltext{Stecken} und \redfilltext{Stab} trösten mich.

\lettrine[lines=2,lraise=0,lhang=1]{\bluetext{D}}%
{\greentext{u}} bereitest vor mir einen \bluefilltext{Tisch} im
\yellowfilltext{\salt{A}n\-ge\-sicht} meiner \redfilltext{Feinde}. \redfilltext{\salt{D}u}
salbest mein \yellowfilltext{Haupt} mit \greenfilltext{Öl} und schenkest mir voll ein.

\lettrine[lines=2,lraise=0,lhang=1]{\goldtext{G}}%
{\redtext{utes}} und \bluefilltext{Barmherzigkeit} werden mir folgen
mein \yellowfilltext{Leben} lang, und ich
werde bleiben im \greenfilltext{Hause} des \rubric{\salt{H}\atc{errn}} immerdar.

\newpage
\ClearWallPaper

\noindent Open\kern-3ptType features\\[1ex]

{\normalsize\irrm
\begin{description}

  \item[\irsc cv01-] Character variants. These fill or color letters. The indices
  are as follows: 1 red-filled, 2 blue-filled, 3 yellow-filled, 4 green-filled,
  5 rubricated, 6 red text, 7 blue text, 8 gold (dark yellow) text, 9 green text.
  Character variant features map to letters as follows: A-Z odd-numbered
  features, cv01-cv51; a-z even-numbered features, cv02-cv52; Æ cv56;
  æ cv57; Œ cv58; œ cv59; Ð cv60; ð cv61; Þ cv62; þ cv63; \& cv64. Small caps
  and the ampersand follow the rules
  for lowercase letters: that is, they can be colored with either Stylistic
  Set or Character Variant features, but they can be filled only with
  Character Variant features.
  
  \item[\irsc dlig] Discretionary ligatures. Unusual ligatures, or alternate forms of
  historical ligatures.

  \item[\irsc hist] Historical forms. Substitutes archaic for modern glyph shapes.

  \item[\irsc hlig] Historical ligatures. Archaic forms of certain ligatures.

  \item[\irsc salt] Stylistic alternates. Substitutes alternate forms of numerous
  glyphs. This duplicates the function of ss10 and ss14 and includes a 
  number of other alternates of standard upper- and lowercase letters,
  titling caps, and small caps.

  \item[\irsc smcp] Small caps. Substitutes rustic capitals for lowercase letters.


  \item[\irsc ss01-ss04] Filled capitals: ss01 red, ss02 blue, ss03 yellow, ss04 green.

  \item[\irsc ss05] Rubricated. Turns all text red.

  \item[\irsc ss06] User color. Text to which ss06 has been applied is black by
  default, and, if an application allows it, can be colored by the user in the
  same way as a standard monochrome font.

  \item[\irsc ss07] Underdotted. Dots are added below letters. In medieval
  manuscripts this indicates deletion.

  \item[\irsc ss10] Alternate y. Substitutes a round-limbed y for the default.

  \item[\irsc ss11] Sentence cap. Substitutes capitals suitable for beginning sentences
  for the default.

  \item[\irsc ss12] Titling caps from caps.

  \item[\irsc ss13] Alternate diacritics. Substitutes alternate forms of\linebreak macron,
  ogonek, and the ur-abbreviation (U+1DE3).

  \item[\irsc ss14] Enlarged minuscules. Enlarged forms of selected lowercase letters:
  {\irit a c d f p}.

  \item[\irsc ss17-ss20] Colored text. Turns all text a color: ss17 red, ss18 blue,
  ss19 gold (dark yellow), ss20 green.

  \item[\irsc swsh] Swash. Swash forms of selected letters: {\irit F M a e i ı n t}.

  \item[\irsc titl] Titling caps. Substitutes titling caps for lowercase letters.
  Use titling caps for all-caps titles and text.

\end{description}
}

\newpage

\noindent Color fonts and software compatibility\\[1ex]

\noindent\normalsize\irrm There are several competing standards for color fonts like Eadui. One is
the COLR/CPAL format, defined by Microsoft, which produces a relatively compact font file
and can be used by all up-to-date web browsers. The file Eadui-CPAL.otf adheres to this
standard: use it if you are mounting Eadui on a web server.

A second standard is based on SVG (Scalable Vector Graphics): it was developed by
Mozilla and Adobe. It produces much larger files---like, for example, the 5.5MB file
Eadui-SVG.otf, which contains both COLR/CPAL and SVG tables and thus will work with
a very wide variety of software.

If you are installing Eadui on your desktop computer, use Eadui-SVG, which will work with
most or all desktop software.

If you are using {Lua\LaTeX} with fontspec, be sure to include the option “Renderer=Harfbuzz” in your
\textbackslash setmainfont command. Other varieties of {\TeX} may treat Eadui as a traditional
monochrome font.

If an application cannot use color fonts, it will treat Eadui like an ordinary monochrome font.
\vfill

\noindent\irit The Eadui font. Copyright © 2011--2024 by Peter S. Baker. Licensed under the Open font
License (https://openfontlicense.org/\kern+0.5pt) and free for all to use. This document was
created with {Lua\LaTeX} and fontspec. The roman font is Irradiant, by Peter S. Baker.



\end{document}