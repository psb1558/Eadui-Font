%&program=xelatex
%&encoding=UTF-8 Unicode
\ifdefined\suppressfontnotfounderror
  \expandafter\let\csname xetex_suppressfontnotfounderror:D\endcsname
    \suppressfontnotfounderror
\else
  \expandafter\let\csname xetex_suppressfontnotfounderror:D\endcsname
    \luatexsuppressfontnotfounderror
\fi
\documentclass[11pt,letterpaper,twoside,openany]{book}
\usepackage[twoside,text={3.8in,502pt},headheight=12pt,headsep=12pt]{geometry}
\usepackage{xltxtra}
\usepackage{fancyhdr}
\usepackage{lettrine}
\usepackage[first=0, last=10, seed=240]{lcg}
\usepackage{color}
\definecolor{stdText}{rgb}{0.289,0.218,0.172}
\definecolor{myRed}{rgb}{0.5,0,0}
\definecolor{myBlue}{rgb}{0,0,0.5}
\definecolor{bkRed}{rgb}{0.905,0.281,0.289}
\definecolor{bkBlue}{rgb}{0.430,0.465,0.688}
\definecolor{bkGreen}{rgb}{0.234,0.445,0.352}
\definecolor{bkGold}{rgb}{0.715,0.520,0.25}
\definecolor{bkMedium}{rgb}{0.312,0.309,0.289}
\definecolor{bkDark}{rgb}{0.195,0.156,0.117}
\definecolor{bkBrown}{rgb}{0.637,0.480,0.398}
\definecolor{bkLtBrown}{rgb}{0.664,0.390,0.352}
\definecolor{bkText}{rgb}{0.371,0.320,0.285}
\definecolor{bkPink}{rgb}{0.604,0.320,0.262}
\definecolor{stdText}{rgb}{0.289,0.218,0.172}
\definecolor{bkOrange}{rgb}{0.809,0.343,0.219}
\definecolor{veryDarkBlue}{rgb}{0.097,0.101,0.109}
\definecolor{GalbaDarkBlue}{rgb}{0.156,0.172,0.258}
\definecolor{GalbaLightBlue}{rgb}{0.310,0.382,0.496}
\definecolor{GalbaRed}{rgb}{0.625,0.203,0.090}
\definecolor{GalbaBrightRed}{rgb}{0.645,0.191,0.063}
\definecolor{VespGreen}{rgb}{0.418,0.461,0.246}
\definecolor{VespRed}{rgb}{0.480,0.113,0.082}
\definecolor{VespTan}{rgb}{0.640,0.363,0.156}
\definecolor{VespGold}{rgb}{0.796,0.621,0.309}
\definecolor{VespBrown}{rgb}{0.414,0.297,0.250}
\definecolor{ForestGreen}{rgb}{0.133,0.445,0.133}
\newcommand{\myColor}[1]{%
  \ifcase#1%\relax
    bkRed%
  \or
    GalbaRed%
  \or
    bkGreen%
  \or
    VespBrown%
  \or
    bkBlue%
  \or
    VespGreen%
  \or
    VespRed%
  \or
    VespTan%
  \or
    ForestGreen%
  \or
    bkPink%
  \else
    GalbaLightBlue%
  \fi
}
\newcommand{\randomColor}{\rand\color{\myColor{\therand}}}
\usepackage{wallpaper}
\newICUfeature{NarrowT}{on}{+ss03}
%
% Some mnemonics to help select Eadui's features.
\newICUfeature{SentenceCaps}{on}{+ss11}
\newICUfeature{Alty}{on}{+ss10}
\newICUfeature{AltCaps}{on}{+ss12}
\newICUfeature{AltDiacritics}{on}{+ss13}
\newICUfeature{LargeMinuscules}{on}{+ss14}
\newICUfeature{Contextual}{on}{+calt}
% End of Eadui mnemonics
%
% N.B. contextuals (calt) should always be on!
\defaultfontfeatures{Mapping=tex-text,Contextual=on}
\setromanfont{Eadui}
%
% Some convenient commands for working with Eadui.
\newcommand{\atc}[1]{{\fontspec[Style=TitlingCaps,AltCaps=on]{Eadui}{#1}}}
\newcommand{\ac}[1]{{\addfontfeatures{AltCaps=on}{#1}}}
\newcommand{\salt}[1]{{\addfontfeatures{Alternate=0}{#1}}}
\newcommand{\saltb}[1]{{\addfontfeatures{Alternate=1}{#1}}}
\newcommand{\altd}[1]{{\addfontfeatures{AltDiacritics=on}{#1}}}
\newcommand{\dlig}[1]{{\addfontfeatures{Ligatures=Rare}{#1}}}
\newcommand{\hlig}[1]{{\addfontfeatures{Ligatures=Historical}{#1}}}
\newcommand{\scap}[1]{{\addfontfeatures{StylisticSet=11}#1}}
\newcommand{\swash}[1]{{\addfontfeatures{Style=Swash}{#1}}}
\newcommand{\hist}[1]{{\addfontfeature{Style=Historic}{#1}}}
\newcommand{\lmin}[1]{{\addfontfeature{LargeMinuscules=on}{#1}}}
\newcommand{\eadui}[1]{{\fontspec{Eadui} #1}}
% End of Eadui commands
%
\newcommand{\opentype}{Open\kern-2pt Type}
\newcommand{\xetex}{X\kern-0.5pt e\kern-2pt Te\kern-1pt X}
\newcommand{\colorcap}[3][-2pt]{\raisebox{#1}[0pt][0pt]{\huge\randomColor #2}}
\renewcommand{\DefaultLraise}{0.2}
\renewcommand{\DefaultLhang}{0.5}
\renewcommand{\DefaultLoversize}{0.1}
%\renewcommand{\DefaultFindent}{0.5em}
\renewcommand{\DefaultNindent}{0em}
\newenvironment{junicode}{\fontspec[NarrowT=on]{Junicode}\normalsize}{}
%
% A "medieval" environment turns on historical forms.
\newenvironment{medieval}{\fontspec[Style=Historic,Alty=on]{Eadui}}{}
%
%
\newcommand{\filltext}[3]{{\fontspec{EaduiFill}\color{#1}#3}%
\llap{\fontspec{Eadui}\color{#2}#3}}
\newcommand{\filltitl}[3]{{\fontspec[Style=TitlingCaps]{EaduiFill}\color{#1}#3}%
\llap{\fontspec[Style=TitlingCaps]{Eadui}\color{#2}#3}}
\newcommand{\junb}[1]{{\fontspec{Junicode}\bfseries{#1}}}
\newcommand{\jun}[1]{{\fontspec{Junicode}{#1}}}
\newcommand{\editem}[1]{{\fontspec{Junicode}\color{myRed}#1}}
\frenchspacing
\linespread{1.04}
\tolerance=1000
\renewcommand{\headrulewidth}{0pt}
\pagestyle{fancy}
\fancyhead{}
\fancyfoot{}


\begin{document}
%\TileSquareWallPaper{6}{parch-light.jpg}
\color{stdText}%\pagecolor{bgColor}
\Large\lettrine[lines=3,lraise=0,lhang=1]{\randomColor E}{}%
%{\textcolor{myBlue}{}}
{\addfontfeatures{Style=TitlingCaps}\randomColor adui is
  named for a scribe who worked at} Christ Church, Canterbury, in the
first half of the eleventh century and signed himself
“\scap{Ead\-uuius cognomento Basan}.”
\colorcap[-1pt]{T}{bkGreen}his \scap{Eadui Basan}
was a leading practitioner of the scribal hand known to paleographers
as style \atc{iv} \scap{E}nglish caroline
minuscule. \colorcap[-3pt]{L}{myRed}\kern-2pt ike caroline minuscules
generally, this one is notable for its legibility; and \scap{E}adui's
work, at its best, possesses a formal beauty that is matched by few
scribes of his time.

\lettrine[lines=2,lraise=0]{\randomColor U}{\randomColor%
  {unlike}} most “medieval” fonts, this one aims not so much to convey
a medieval feeling as to reproduce a great scribe's style with
fidelity, avoiding the machined look of the digital font in favor of
the irregular and uneven look of a handmade thing. Lines are rarely straight
and often a little wavery, serifs
differ in shape from one glyph to another, letters do not always sit
cleanly on the baseline, and contextual variants vary some of the most
common letters.

\lettrine[lines=2,lraise=0]{\randomColor \salt{E}}{\randomColor adui}
the font (like the scribe)
  employs a variety of capitals for various purposes. The default
  capitals in the font are those he used for what are now called
  “drop caps.” Stylistic Set 11 shifts to the kind of capitals he used
  to begin sentences. These are often identical with Titling Caps,
  which are suitable for rubrics, titles, and emphasis.

\newpage
\noindent Historic mode (hist or ss16):\\[1ex]

\begin{medieval}
\huge\color{bkMedium}\lettrine[lines=3,lraise=0,lhang=1]{\randomColor B}%
{}{\scshape\color{bkBrown} eat᷒ uir qͥ non abiit in
consilio impiorum} \& in uia peccatorum non stetit \& in cathedra
pestilentiae n̄ \kern-2pt sedi\swash{t}.

\lettrine[lhang=1]{\randomColor S\hspace{3pt}}{}ed in lege dn̄i fuit uoluntas eius
\& in lege eius meditabitur die ac no\hlig{ct}\swash{e}.

\lettrine[lhang=1]{\randomColor E\hspace{3pt}}{}t erit tanquam lignum quod
plantatum est secus decursus aquarum quod fru\hlig{ct}um suum dabit in
tempore suo.

\lettrine[lhang=1]{\addfontfeatures{Ligatures=Rare}\randomColor ET\hspace{3pt}}{}foliū
ei᷒ \kern-1pt non defluet \& omnia quęcumq: faciet prosperabu\swash{n}tᷣ.

\lettrine[lhang=1]{\randomColor N\hspace{3pt}}{}on sic impi\swash{i} \salt{n}on sic sed
tanquam pul\-uis quem ꝓicit uent᷒ a facie terr\altd{ę}.

\lettrine[lhang=1,nindent=0.2em]{\randomColor I\hspace{3pt}}{}deo non resurgu\dlig{nt} impii in
iudicio neq: peccat\hlig{or}es in consilio iustorum.

\lettrine[lhang=1]{\addfontfeatures{AltCaps=on}\randomColor Q\hspace{3pt}}{}\altd{m̄} nouit
dn̄s uiam iustorum \& it͛ impioꝝ ꝑibi\swash{t}.

\end{medieval}


\newpage
\noindent With historic ligatures:\\[0.25ex]

\begin{medieval}
\Large\noindent{\randomColor\addfontfeatures{Style=TitlingCaps}%
  \salt{G}\kern+1.5pt loriosissi\salt{m}o regi C\salt{e}oluulfo Baeda
  f\salt{a}mul᷒ Christi et presbyter}

\addfontfeatures{Ligatures=Historical}\color{bkDark}
\lettrine[lines=3,lraise=0]{\randomColor H}{istoriam}
\addfontfeature{SentenceCaps=on}
gentis Anglorum ecclesiasticam, quam nuper edi\-deram,
libentissime tibi desideranti, rex, et prius ad legendum ac probandum
transmisi, et nunc ad transscribendum ac plenius ex tempore meditandum
retransmitto; satisque studium tuae sinceritatis amplector, quo non
solum audiendis scripturae sanctae uerbis aurem sedulus accommodas,
uerum etiam noscendis priorum gestis siue dictis, et maxime nostrae
gentis uirorum inlustrium, curam uigilanter impendis. Siue enim
historia de bonis bona referat, ad imitandum bonum auditor sollicitus
instigatur; seu mala commemoret de prauis, nihilominus religiosus ac
pius auditor siue lector deuitando quod noxium est ac peruersum, ipse
sollertius ad exsequenda ea, quae bona ac Deo digna esse cognouerit,
accenditur. Quod ipsum tu quoque uigilantissime deprehendens,
historiam memoratam in notitiam tibi simul et eis, quibus te regendis
diuina praefecit auctoritas, ob generalis curam salutis latius
propalari desideras. Vt autem in his quae scripsi ƚ tibi, ƚ
ceteris auditoribus siue lectoribus huius historiae occasionem
dubitandi subtraham, quibus haec maxime auctoribus didicerim, breuiter
intimare curabo.
\end{medieval}


\newpage
\noindent Rustic capitals (small caps):\\[0.25ex]

\begin{medieval}
\lettrine[lines=3,lraise=0]{\randomColor B}{\randomColor rittania oceani insula,}\scshape\color{bkText}
\addfontfeatures{AltCaps=on} \ cui quondā
\salt{Al}bion nomen fuit, inter septentrionē \kern-2pt \& occidentē locata
est, Germanię, Gallię, Hispanię, maximis \salt{E}uropę partib᷒, multo
interuallo aduersa. Quae per miliapas\salt{s}uū .dccc. in Boreā
longa, latitudinis hab\& milia .cc., exceptis dumtaxat pro\-lixiorib᷒
diuersorū promontoriorū tractib᷒, quibus efficitᷣ, ut circuit᷒ ei᷒
quadragies octies .lxxv. milia conpleat. Habet a meridie Galliā
Belgicā, cui᷒ proximū lit᷒ transmeantib᷒ aperit ciuitas, quae dicitᷣ
Rutubi port᷒, a gente \salt{A}nglorū nunc corrupte Reptacaestir uocata,
interposito mari a Ges\salt{s}oriaco Morynorū gentis litore proximo,
traiectu miliū .l., siue, ut quidā scripsere, stadiorum .ccccl. A
tergo autem, unde Oceano infinito patet, Orcadas insulas habet. Opima
frugib᷒ atque arborib᷒ insula, \& alendis apta pecorib᷒ ac
iumentis; uineas \&iam quibusdā in locis germinans; sed \& auiū
ferax terra mariq: generis diuersi; fluuiis quoq: multū piscosis ac
fontib᷒ praeclara copiosis, \& quidē praecipue issicio abundat, \&
anguilla.
\end{medieval}

\newpage
\noindent Modern languages:\\[0.25ex]

\huge\color{bkMedium}%
\lettrine[lines=3,lraise=0,lhang=0.5,findent=-25pt,nindent=25pt]%
{\randomColor H}{\randomColor eureux}
l'homme qui ne marche pas selon le conseil des méchants,
\colorcap[-1pt]{Q}{bkBluea}ui
ne s'arrête pas sur la voie des pécheurs,
\colorcap[-1.5pt]{E}{bkGreen}t qui ne s'assied pas en
compagnie des moqueurs,

\lettrine[lines=2,lraise=0,lhang=1]{\randomColor M}%
{\randomColor ais}
qui trouve son plaisir dans la loi de l'Éternel,
\colorcap[-1.5pt]{E}{bkPinka}t qui la médite
jour et nuit!

\lettrine[lines=2,lraise=0,lhang=1]{\randomColor I\kern+5pt}%
{\randomColor l est}
comme un arbre planté près d'un cou\-rant d'eau,
\colorcap[-1pt]{\salt{Q}}{bkOrange}ui donne son
fruit en sa saison,
\colorcap[-1.5pt]{E}{bkBlue}t dont le feuillage ne se flétrit point:
\colorcap{T}{bkGold}out ce
qu'il fait lui réussit.

\lettrine[lines=2,lraise=0,lhang=1]{\randomColor I\kern+5pt}%
{\randomColor l n'en}
est pas ainsi des méchants:
\colorcap[0pt]{I}{bkBlue}\kern+1pt ls sont comme la paille que le
vent dissipe.

\lettrine[lines=2,lraise=0,lhang=1]{\randomColor C}%
{\randomColor 'est}
pourquoi les méchants ne résistent pas au jour du jugement,
\colorcap[-1pt]{N}{bkBluea}i
les pécheurs dans l'assemblée des justes;

\lettrine[lines=2,lraise=0,lhang=1]{\randomColor C}%
{\randomColor ar}
l'Éternel connaît la voie des justes,
\colorcap[-1.5pt]{E}{bkPink}t la voie des pécheurs mène
à la ruine.

\newpage

\huge\color{bkMedium}\lettrine[lines=2,lraise=0,lhang=1]%
{\randomColor\salt{D}}{\textcolor{bkOrange}{er Herr}} ist mein
\salt{H}irte, mir wird nichts mangeln.

\lettrine[lines=2,lraise=0,lhang=1]{\randomColor \salt{E}}%
{\randomColor r} weidet mich auf einer grünen Aue und
führet mich zum frischen Wasser.

\lettrine[lines=2,lraise=0,lhang=1]%
{{\randomColor E}}{\randomColor r}
erquicket meine Seele. Er führet mich auf rechter Straße um seines
Namens willen.

\lettrine[lines=2,lraise=0,lhang=1]{\randomColor U}%
{\randomColor nd} ob ich schon wanderte im finstern Tal,
fürchte ich kein Unglück;
denn du bist bei mir, dein Stecken und Stab trösten mich.

\lettrine[lines=2,lraise=0,lhang=1]{\randomColor D}%
{\randomColor u} bereitest vor mir einen Tisch im
\salt{A}n\-ge\-sicht meiner Feinde. \salt{D}u
salbest mein Haupt mit Öl und schenkest mir voll ein.

\lettrine[lines=2,lraise=0,lhang=1]{\randomColor G}%
{\randomColor utes} und Barmherzigkeit werden mir folgen
mein Leben lang, und ich
werde bleiben im Hause des \salt{H}\atc{errn} immerdar.

\newpage

\noindent\randomColor abcdefghijklmnopqrstuvwxyzæðþ

\noindent\randomColor ABCDEFGHIJKLMNOPQRSTUVWXYZ\\
ÆÐÞ

\noindent\randomColor{\addfontfeatures{Style=TitlingCaps}
  abcdefghijklmnopqrstuvwxyzæðþ}

\noindent\randomColor\textsc{abcdefghijklmnopqrstuvwxyzæðþ fl fs}

\noindent\randomColor 1234567890 (!@\#\$\%\&*-+=?\,,\,.\,:\,;\,“\,”\,‘\,’)

\noindent\randomColor [£\char"20AC ¢¥§¶•ªº\,\{\,\}\,©®¼½¾¡¿]

\noindent\randomColor đ Œ œ ſ ƚ ƿ ę \altd{ę} \char"F1E1\ \char"F160\ \char"F161\ \char"F1EA

\noindent\randomColor uidet\char"1DE3\ ei\char"1DD2\
ment\char"035B\ g\char"0365\ n\char"0366\ \char"A751ficere
\char"A753fug\char"1DD2\ o\char"A75Bdines meo\char"A75D

\noindent\randomColor offer first flat afflict office ſt\&it facit \dlig{ſtıpendıū
  ET ſunt} \swash{M}ater \hist{gaudeamus furore epistola
ere\hlig{ct}us exercitus ergo euge forceps \char"A751\swash{n}untia\swash{t}}

\newpage

\noindent\Large\color{black} I “Ask Jeff” or ‘Ask Jeff’. Take the chef d’œuvre! Two of [of] (of) ‘of’ “of” of? of! of*. Ydes, Yffignac and Ygrande are in France: so are Ypres, Les Woëvres, the Fôret de Wœvres, the Voire and Vauvise. Yves is in heaven; D’Amboise is in jail. Lyford’s in Texas \& L’Anse-aux-Griffons in Québec; the Łyna in Poland. Yriarte, Yciar and Ysaÿe are at Yale. Kyoto and Ryotsu are both in Japan, Kwikpak on the Yukon delta, Kvæven in Norway, Kyulu in Kenya, not in Rwanda.… Walton’s in West Virginia, but «Wren» is in Oregon. Tlálpan is near Xochimilco in México. The Zygos \& Xylophagou are in Cyprus, Zwettl in Austria, Fænø in Denmark, the Vøringsfossen and Værøy in Norway. Tchula is in Mississippi, the Tittabawassee in Michigan. Twodot is here in Montana, Ywamun in Burma. Yggdrasil and Ymir, Yngvi and Vóden, Vídrið and Skeggjöld and Týr are all in the Eddas. Tørberget and Våg, of course, are in Norway, Ktipas and Tmolos in Greece, but Vázquez is in Argentina, Vreden in Germany, Von-Vincke-Straße in Münster, Vdovino in Russia, Ytterbium in the periodic table. Are Toussaint L’Ouverture, Wölfflin, Wolfe %,Miłosz
and Wū % Wŭ
 all in the library?
1510–1620, 11:00 pm, and the 1980s are over. 
Ergänzt von Typefacts: 
Ist da „Jemand“? „Volker?“ – „Wolf“. „Anna?“ – „Yvonne“. „Torsten fragte: ‚Vladimir?‘, später rief er ‚Wolf‘ und ‚Theresa‘, dann ‚Andreas‘ und ‚Yvonne‘“. Eleganter: Ist da »Jemand«? »Volker?« – »Wolf«. »Anna?« – »Yvonne«. »Torsten fragte: ›Vladimir?‹, später rief er ›Wolf‹ und ›Theresa‹, dann ›Andreas‹ und ›Yvonne‹«.
\end{document}
